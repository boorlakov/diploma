\hypersetup{
  pdfauthor={Бурлаков В.С.}, % Автор
  pdftitle={T5FGA: T5 For Gaming Applications}, % Заголовок
  pdfsubject={Выпускная квалификационная работа}
}

\onehalfspacing

\definecolor{mygreen}{rgb}{0,0.6,0}
\definecolor{mygray}{rgb}{0.5,0.5,0.5}
\definecolor{mymauve}{rgb}{0.58,0,0.82}
\lstloadlanguages{Python}
\lstset{
  language=Python,
  keywordstyle={\bfseries \color{blue}},
  captionpos=b,
  tabsize=4,
  basicstyle=\footnotesize,
  breaklines=true,
  breakatwhitespace=true,
  extendedchars=true,
  backgroundcolor=\color{white},
  commentstyle=\color{mygreen},
  escapeinside={\%*}{*)},
  morekeywords={
      self, required, from, import,
      def, return, if, else, elif,
      for, in, while, True, False,
      None, with, as, try, except,
      finally, class, pass, break,
      continue, lambda, yield, assert,
      async, await, nonlocal, global,
      del, not, or, and, is, in,
      not in, is not, True, False, None,
      set_format, save, batch_encode_plus,
      train, evaluate, load_data
    },
  stringstyle=\color{mymauve}
}

% \usepackage{graphicx}
\graphicspath{ {./figures/} }

% Параметры названия для иллюстраций (Рисунок 1 - <название>)
\DeclareCaptionLabelFormat{PictureCaptionFormat}{Рисунок {#2}}
\captionsetup[figure] {
  labelformat=PictureCaptionFormat,
  skip=0pt,
  format=hang,
  justification=raggedright,
  labelsep=endash
}

% Параметры названия для таблиц (Таблица 1 - <название>)
\DeclareCaptionLabelFormat{TableCaptionFormat}{Таблица {#2}}
\captionsetup[table] {
  labelformat=TableCaptionFormat,
  skip=0pt,
  format=hang,
  justification=raggedright,
  singlelinecheck=off,
  labelsep=endash
}

\DeclareCaptionLabelFormat{ListingCaptionFormat}{Листинг {#2}}
\captionsetup[lstinputlisting] {
  labelformat=ListingCaptionFormat,
  skip=0pt,
  format=hang,
  justification=raggedright,
  labelsep=endash
}