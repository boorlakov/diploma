\hypersetup{
  pdfauthor={Burlakov V.S.},
  pdftitle={Research on the use of generative neural network models in the development of a dialogue agent creation system},
  pdfsubject={Vladimir Burlakov's Bachelor Thesis},
  pdfkeywords = {machine learning, natural language processing, generative AI, artificial intellegence, game development, T5}
}

\onehalfspacing

\definecolor{mygreen}{rgb}{0,0.6,0}
\definecolor{mygray}{rgb}{0.5,0.5,0.5}
\definecolor{mymauve}{rgb}{0.58,0,0.82}
\lstloadlanguages{Python}
\lstset{
  language=Python,
  keywordstyle={\bfseries \color{blue}},
  captionpos=b,
  basicstyle=\footnotesize,
  breaklines=true,
  breakatwhitespace=true,
  extendedchars=true,
  backgroundcolor=\color{white},
  commentstyle=\color{mygreen},
  escapeinside={\%*}{*)},
  stringstyle=\color{mymauve},
  numbers=left,
  numberstyle=\tiny\color{gray},
  tabsize=4,  % adjust to your preference
}

\graphicspath{ {./figures/} }

% Параметры названия для иллюстраций (Рисунок 1 - <название>)
\DeclareCaptionLabelFormat{PictureCaptionFormat}{Рисунок {#2}}
\captionsetup[figure] {
  labelformat=PictureCaptionFormat,
  skip=0pt,
  format=hang,
  justification=raggedright,
  labelsep=endash
}

% Параметры названия для таблиц (Таблица 1 - <название>)
\DeclareCaptionLabelFormat{TableCaptionFormat}{Таблица {#2}}
\captionsetup[table] {
  labelformat=TableCaptionFormat,
  skip=0pt,
  format=hang,
  justification=raggedright,
  singlelinecheck=off,
  labelsep=endash
}

\DeclareCaptionLabelFormat{ListingCaptionFormat}{Листинг {#2}}
\captionsetup[lstinputlisting] {
  labelformat=ListingCaptionFormat,
  skip=0pt,
  format=hang,
  justification=raggedright,
  labelsep=endash
}

\usetikzlibrary{matrix,chains,positioning,decorations.pathreplacing,arrows,arrows.meta}
\makeatletter
\def\BState{\State\hskip-\ALG@thistlm}
\makeatother
