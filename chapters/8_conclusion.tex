В рамках данного исследования была поставлена цель исследования использования генеративных нейросетевых моделей и разработки эффективной диалоговой модели, способной генерировать качественные ответы на основе образа неигрового персонажа и контекста диалога. Был использован и проанализирован специально созданный для исследования набор данных DNDD (Dungeon \& Dragons Dialogues). Также этот набор данных подготовлен специально для эмуляции диалогов в играх. В процессе исследования был изучен метод обучения моделей искусственных нейронных сетей с учителем, архитектура транслятор и было исследовано влияние различных параметров, таких как скорость обучения и планировщик скорости обучения на процесс обучения.

На основе проведенных экспериментов можно сделать вывод о наилучшем выборе параметров для обучения модели. Для модели Flan-T5 был выявлен оптимальный планировщик скорости обучения - константный планировщик, а оптимальное значение скорости обучения составляет $9 \times 10^{-4}$. Это сочетание показало лучшие результаты по функциям ошибок и метрикам на представленных наборах данных.

В целом, полученные результаты демонстрируют возможность обучения эффективной диалоговой модели на доступных вычислительных ресурсах. Дальнейшие исследования и улучшения в области диалоговых моделей могут привести к еще более точным и качественным результатам.