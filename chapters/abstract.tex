Отчет 82 с., 5 ч., 30 рис., 3 табл., 33 источника, 3 прил.

МАШИННОЕ ОБУЧЕНИЕ, ИСКУССТВЕННЫЕ НЕЙРОННЫЕ СЕТИ, ОБРАБОТКА ЕСТЕСТВЕННОГО ЯЗЫКА, ГЕНЕРАЦИЯ ЕСТЕСТВЕННОГО ЯЗЫКА, ТРАНСФОРМЕР, ДИАЛОГОВЫЕ СИСТЕМЫ.

$ $

В этой работе исследуется использование генеративных нейросетевых моделей при создании диалоговых агентов для игровых приложений. Также изучается создание набора данных DNDD и процесс обучения с использованием больших языковых моделей. В исследовании учитывается текстовое изображение неигровых персонажей, чтобы создать более точный портрет неигрового персонажа. Рассматриваются различные аспекты процесса моделирования, включая сбор и предварительную обработку данных, постановку задач, выбор и оптимизацию модели, а также реализацию демо приложения для создания диалоговых агентов для разработчиков видеоигр. Исследование демонстрирует потенциал методов глубокого обучения в улучшении взаимодействия человека с компьютером и создании более увлекательных игровых впечатлений.

$ $

По теме работы было выступление на XVI Всероссийской научной конференции молодых ученых «НАУКА. ТЕХНОЛОГИИ. ИННОВАЦИИ» и была опубликована статья \cite{nti-paper}.