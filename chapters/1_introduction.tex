В настоящее время разработка диалоговых систем является важным направлением машинного обучения, в условиях растущей автоматизации взаимодействия человека с компьютером. Такие технологии могут быть использованы в различных отраслях, включая игровую. Недавние достижения в глубоком обучении позволяют создавать диалоговые модели, которые способны обрабатывать естественный язык и предоставлять пользователю качественные ответы на запросы. Для успешного обучения таких моделей необходимы качественные наборы данных, содержащие диалоги между участниками. В данной работе рассматривается процесс создания и анализа набора данных DNDD (Dungeon \& Dragons Dialogues) с использованием текущих больших языковых моделей для обучения диалоговой модели, основанный на сборе и предобработке данных из различных источников с учетом образа неигровых персонажей. Целью работы является создание диалоговой модели, которая будет способна генерировать ответы на основе образа неигрового персонажа и контекста диалога. Такая модель позволить улучшить игровой опыт игроков, сделать этот опыт более иммерсивным. В данной работе рассматривается процесс подготовки набора данных для обучения модели, формулирование задачи для моделирования, поиск оптимальной модели и параметров для успешного и эффективного обучения.