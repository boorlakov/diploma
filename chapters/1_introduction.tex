% FROM FIRST PRACTICE
% В современных условиях растущей автоматизации взаимодействия человека с компьютером, создание эффективных диалоговых систем остается одним из важных направлений развития искусственного интеллекта. Такие технологии могут помочь во многих индустриях, в том числе игровой индустрии. Новейшие технологии, использующие глубокое обучение, позволяют создавать диалоговые модели, которые способны эффективно обрабатывать естественный язык и предоставлять пользователю качественные ответы на запросы. Для эффективного обучения диалоговых моделей необходимо обладать качественным набором данных, содержащим диалоги между участниками. В данной работе было решено расширить концепцию диалогового датасета путем включения описания личности участников разговора. Это имеет большое значение при использовании в игровой индустрии.В данном отчете рассматривается процесс создания и анализа датасета DNDD (Dungeon \& Dragons Dialogues) для обучения диалоговой модели, основанный на сборе и предобработке данных из различных источников.

Разработка диалоговых моделей является активно развивающейся областью машинного обучения. Использование таких моделей имеет широкий спектр применений, включая чат-ботов, системы FAQ, и различные другие системы, где взаимодействие с пользователем через естественный язык играет важную роль. В игровой индустрии диалоговые модели имеют особое значение, поскольку они способны создавать реалистичные и интерактивные диалоги с неигровыми персонажами, улучшая игровой опыт. Качественные диалоговые модели способны улучшить игровой опыт, создавая более привлекательные и погружающие виртуальные миры.

Целью данной работы является создание эффективной диалоговой модели, способную генерировать качественные ответы на основе образа неигрового персонажа и контекста диалога, обеспечивая более реалистичные и интерактивные диалоги с неигровыми персонажами, на основе датасета DNDD (Dungeon \& Dragons Dialogues), специально созданного для данного исследования. В данной работе рассматриваются подготовка датасета для обучения модели, формулирование задачи для моделирования, поиск оптимальной модели и параметров, необходимых для успешного и эффективного обучения.