% FROM FIRST PRACTICE
% В заключение, проанализированный набор данных дает ценную информацию о характеристиках взаимодействия NPC в видеоиграх. Высокий процент ответов NPC в наборе данных показывает важность этих взаимодействий в игровом процессе. Представленные цифры показывают, что ответы NPC содержат больше контекстной информации, что приводит к более детальным и более качественным диалогам. Дальнейшие исследования в этой области могут привести к разработке более совершенных систем искусственного интеллекта для NPC в видеоиграх, что в конечном итоге улучшит общий игровой опыт для игроков.

В рамках данного исследования была поставлена цель разработки эффективной диалоговой модели, способной генерировать качественные ответы на основе образа неигрового персонажа и контекста диалога в игровой индустрии. Был использован специально созданный для исследования датасет DNDD (Dungeon \& Dragons Dialogues). И подготовлен специально для эмулирования диалогов в играх. В процессе экспериментов были рассмотрены различные параметры и планировщики скорости обучения.

На основе проведенных экспериментов можно сделать вывод о наилучшем выборе параметров для обучения модели. Для модели Flan-T5 был выявлен оптимальный планировщик скорости обучения - константный планировщик, а оптимальное значение скорости обучения составляет $9 \times 10^{-4}$. Это сочетание показало лучшие результаты по функциям ошибок и метрикам на представленных наборах данных.

Однако, следует отметить, что введенная сложность задачи диалоговой моделирования в игровой индустрии, где ответы зависят от различных условий и контекста диалога, может быть причиной низких значений метрик Exact Match и MAUVE. Оценка Exact Match грубо оценивает результат генерации, причем даже переформулировка фразы может привести к низким значениям.

В целом, полученные результаты демонстрируют возможность обучения эффективной диалоговой модели на доступных вычислительных ресурсах. Однако, дальнейшие исследования и улучшения в области диалоговых моделей могут привести к более точным и качественным результатам.