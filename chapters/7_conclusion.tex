В рамках данной работы исследовано использование генеративных нейросетевых моделей для разработки эффективной модели диалога, способной генерировать качественные ответы на основе образа неигрового персонажа и контекста диалога. Для этого был проанализирован специально созданный набор данных DNDD (Dungeon \& Dragons Dialogues), подготовленный специально для эмуляции диалогов в играх.

Для работы был изучен метод обучения моделей искусственных нейронных сетей с учителем с акцентом на архитектуру трансформера. Было исследовано влияние различных гиперпараметров, таких как скорость обучения и планировщик скорости обучения, на процесс обучения. Были оценены различные методы оптимизации производительности модели, включая квантование, FlashAttention и BetterTransformers.

На основе экспериментов была определена наилучшая комбинация гиперпараметров для обучения модели Flan-T5. Константный планировщик оказался оптимальным планировщиком скорости обучения с оптимальной скоростью обучения $9 \times 10^{-4}$. Эта комбинация дала наилучшие результаты для функций ошибок и метрик в представленных наборах данных.

Кроме того, в качестве практического применения исследования было разработано демонстрационное приложение, которое позволяет разработчикам видеоигр легко создавать диалоговых агентов, используя обученную модель. Это может значительно улучшить игровой процесс, добавив реалистичные и увлекательные диалоги между NPC и игроками. В целом, работа подчеркивает потенциал использования моделей генеративных нейронных сетей для создания диалоговых агентов в игровой индустрии.

Результаты показывают, что можно обучить эффективную диалоговую модель на доступных вычислительных ресурсах. Дальнейшие исследования и улучшения диалоговых моделей, такие как разработка более качественных показателей оценки качества, могут привести к еще более точным и качественным результатам.